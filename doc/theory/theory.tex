\documentclass[11pt, oneside, a4paper]{article}   	
\usepackage{graphicx}					

\title{Multicomponent, multiphase flow in porous media}
\author{Chris Green}
\date{\today}	

\begin{document}
\maketitle

\section{Governing equations}

\subsection{Definitions and nomenclature}

\subsubsection*{Phase}


\subsubsection*{Component}

\subsubsection*{Notation and nomenclature}

In all following equations, the subscript $\alpha$ refers to the phase, while the superscript $\kappa$ refers to the component.

\begin{table}[h]
\begin{tabular}{llll}
$P_{\alpha}$ & Phase pressure & $\phi$ & Porosity \\
$T$ & Temperature & $K$ & Permeability \\
$S_{\alpha}$ & Phase saturation & $k_{r \alpha}$ & Relative permeability \\
$P_c$ & Capillary pressure & $\rho_{\alpha}$ & Phase density \\
$\mu_{\alpha}$ & Phase viscosity & $\mathbf{g}$ & Gravity \\
$X_{\alpha}^{\kappa}$ & Mass fraction of component $\kappa$ in phase $\alpha$ & & \\
$q_{\alpha}^{\kappa}$ & Source/sink term of $\kappa$ in phase $\alpha$ & &
\end{tabular}
\label{default}
\end{table}

\subsection{Mass balance equation}

Conservation of mass of of each component $\kappa$ gives the following balance equation for each component (for an isothermal system)

\begin{equation}
\phi \frac{\partial}{\partial t} \left(\sum_{\alpha} \rho_{\alpha} X_{\alpha}^{\kappa} S_{\alpha} \right) =  \sum_{\alpha} \nabla \cdot \left\{\frac{K k_{r \alpha} \rho_{\alpha} X_{\alpha}^{\kappa}}{\mu_{\alpha}} \left( \nabla P_{\alpha} - \rho_{\alpha} \mathbf{g} \right) \right\} + \sum_{\alpha} q_{\alpha}^{\kappa}.
\label{eq:massbalance}
\end{equation}

The number of unknown variables can be reduced using fundamental relationships and constitutive models. The phase saturations must sum to unity (assuming that the entire pore space is occupied)
\begin{equation}
\sum_{\alpha} S_{\alpha} = 1,
\end{equation}
while the sum of mass fraction components in each phase must be unity by definition
\begin{equation}
\sum_{\kappa} X_{\alpha}^{\kappa} = 1.
\end{equation}

Relative permeability is calculated as a function of saturation (using a prescribed relationship). The pressure of each phase is related by capillary pressure, defined as 
\begin{equation}
P_c = P_n - Pw,
\end{equation}
where the subscripts $n$ and $w$ refer to the non-wetting and wetting phases, respectively. Like relative permeability, capillary pressure can be calculated as a function of saturation using prescribed relationships.


\end{document}  